\documentclass{article}

\usepackage{amsmath} % For align and other math environments
\usepackage{setspace} % For double spacing
\usepackage{mathtools} % amsmath extensions
\usepackage{amsfonts} % math fonts
\usepackage{graphicx} % for images
\usepackage{listings} % for code blocks

\graphicspath{{ ./images/ }}

\begin{document}


\section{Appendix}
TOFIX: Add term $X_{t}$ to the OU process, it should be $X_{t + 1} = X_{t} + -\theta \cdots$
\singlespacing

Code for generating a sample Wiener process:

\begin{lstlisting}
import matplotlib.pyplot as plt
import seaborn as sns
import pandas as pd
import numpy as np

NUM_PATHS = 10
SAMPLES = 1000

MU = 0
SIGMA = 1

rng = np.random.default_rng()

def compute_wiener() -> list[float]:
    value = 0
    wiener = []

    for _ in range(SAMPLES):
        s = rng.normal(MU, SIGMA)
        value += s

        wiener.append(value)

    return wiener


df_data = {}
for i in range(NUM_PATHS):
    path = compute_wiener()
    df_data[f'WPath {i}'] = path

plt.figure(figsize=(10, 7))

data = pd.DataFrame(df_data)
sns.lineplot(data=data)

plt.legend([], [], frameon=False)
plt.show()
\end{lstlisting}

Code for generating a sample naive Ornstein-Uhlenbeck process:
\begin{lstlisting}
import matplotlib.pyplot as plt
import seaborn as sns
import pandas as pd
import numpy as np

NUM_PATHS = 2
SAMPLES = 200
THETA = 0.2
MEAN = 0

MU = 0
SIGMA = 1

rng = np.random.default_rng()

def compute_wiener() -> list[float]:
    value = 0
    wiener = []

    for _ in range(SAMPLES):
        mean_rev_term = -THETA * (value - MEAN)
        s = mean_rev_term + rng.normal(MU, SIGMA)
        value += s

        wiener.append(value)

    return wiener


df_data = {}
for i in range(NUM_PATHS):
    path = compute_wiener()
    df_data[f'WPath {i}'] = path

data = pd.DataFrame(df_data)

long_form = pd.melt(data.reset_index(), id_vars='index', var_name='Path', value_name='Value')
long_form.rename(columns={'index': 'Time'}, inplace=True)

plt.figure(figsize=(10, 7))
sns.lineplot(data=long_form, x='Time', y='Value', hue='Path')

plt.legend([], [], frameon=False)
plt.show()
\end{lstlisting}

Note that this is just a draft, code will be refactored and common functions will be declared to shorten the appendix and remove redundancies.

\end{document}
