\documentclass{article}
\usepackage{amsmath} % For align and other math environments
\usepackage{setspace} % For double spacing

% notes:
% https://hudsonthames.org/caveats-in-calibrating-the-ou-process/

\begin{document}
\doublespacing

\section{Brownian Motion and the Wiener Process}

Consider a stochastic process, where per each iteration, we add a $\Delta$ to the previous value,
this $\Delta$ is distributed by a normal distribution with a $mu$ of zero, and these are independent
stochastic events this is called the Wiener process. In discrete time, we can define the Wiener process
with the following equation

\begin{gather*}
    X_{t + 1} = X_{t} + \mathcal{N}(0, \sigma)
\end{gather*}

Where $X_{t + 1}$ is the next value of the process (under discrete time), $\mathcal{N}(0, \sigma)$
is a normally distributed drift variable with the variance $\sigma$.

Intuitively, the expected value, i.e. the limit of this process as it approaches infinity, should
be zero, since the normal distribution is symmetric.

We know that the the difference between any two consecutive discrite time-points in the Wiener process
will result in a normally distributed variable $X_{t + 1} - X_{t} = \mathcal{N}(0, \sigma)$, therefore,
we can 

\begin{gather*}

\end{gather*}

\end{document}
